\documentclass{article}
\usepackage[romanian]{babel}
\usepackage{geometry}
\usepackage{amsmath}
\usepackage{graphicx}
\usepackage{tcolorbox}
\usepackage{tikz}
\usepackage{pgfplots}
\usepackage{hyperref}
\usepackage{amssymb} 
\usepackage{pifont}  
\usepackage{enumitem} 
\usepackage{amsmath}
\usepackage{amssymb}
\usetikzlibrary{shapes, arrows.meta, positioning}

\title{Învățare automată - Temă S10}
\author{Balan Călin (3B1)}
\date{6 decembrie 2024}
\begin{document}
\maketitle

\section*{Problema 0.164A / pag. 254}

Cunoaștem \( f \) funcție derivabilă, \( f : \mathbb{R}^d \to \mathbb{R} \). \newline
Regula de actualizare este: \( x_i^{(k+1)} \leftarrow x_i^{(k)} - learningRate \cdot \frac{\partial}{\partial x_i} f(x^{(k)}) \)

\noindent \textbf{a}.

\( f(x) = 4x^2 - 2x + 1 \)

\( learningRate = 0.1 \)

\(\frac{\partial}{\partial x}f(x) = \frac{\partial}{\partial x}(4x^2-2x+1) = 8x-2\)

\( x^{(1)} = 1 \)

Pentru \( k = 0 \):

\( \frac{\partial}{\partial x} f(x^{(0)}) = 6 \)

\( x^{(1)} = x^{(0)} - learningRate \cdot 6 = 1 - 0.1 \cdot 6 = 0.4 \)

\( f(x^{(1)}) = f(0.4) = 0.84 \)

Pentru \( k = 1 \):

\( \frac{\partial}{\partial x} f(x^{(1)}) = 1.2 \)

\( x^{(2)} = x^{(1)} - learningRate \cdot 1.2 = 0.4 - 0.1 \cdot 1.2 = 0.28 \)

\( f(x^{(2)}) = f(0.28) = 0.7536 \)

Pentru \( k = 2 \):

\( \frac{\partial}{\partial x} f(x^{(2)}) = 0.24 \)

\( x^{(3)} = x^{(2)} - learningRate \cdot 0.24 = 0.28 - 0.1 \cdot 0.24 = 0.256 \)

\( f(x^{(3)}) = f(0.256) = 0.7501 \)

\noindent \textbf{b}.

\( f(x_1, x_2) = x_1^2 + sin(x_1 + x_2) + x_2^2 \)

\( x^{(k)} = (x_1^{(k)}, x_2^{(k)}) \)

\( x_1^{(k+1)} \leftarrow x_1^{(k)} - learningRate \cdot \frac{\partial}{\partial x_1} f(x_1^{(k)}, x_2^{(k)}) \)

\( x_2^{(k+1)} \leftarrow x_2^{(k)} - learningRate \cdot \frac{\partial}{\partial x_2} f(x_1^{(k)}, x_2^{(k)}) \)

\( \frac{\partial}{\partial x_1} f(x_1^{(k)}, x_2^{(k)}) = 2x_1 + cos(x_1 + x_2) \)

\( \frac{\partial}{\partial x_2} f(x_1^{(k)}, x_2^{(k)}) = 2x_2 + cos(x_1 + x_2) \)

\( f : [-4, 4] \to [-4, 4] \)

\textbf{i}. Primii 10 pași făcuți de \textbf{GD} începând cu \( (x_1^{(0)}, x_2^{(0)}) = (3, -3) \) și \( learningRate = 0.4 \):

\begin{center}
    \includegraphics[width=0.5\textwidth]{1.png}
\end{center}

\textbf{ii}. Primii 10 pași făcuți de \textbf{GD} începând cu \( (x_1^{(0)}, x_2^{(0)}) = (3, -3) \) și \( learningRate = 0.8 \):

\begin{center}
    \includegraphics[width=0.5\textwidth]{2.png}
\end{center}

Se observă faptul că punctele converg către o valoare de minim. Mai mult, \( learningRate \)-ul configurează dimensiunea "pașilor" cu care covergența are loc.

\section*{Problema 1.34 / pag. 332}

Vom schimba setul de date inițial cu cel de la \textbf{Problema 4.6 / pag. 491}.

\noindent Astfel, vom face următoarele translări pentru a ne ajuta în rezolvarea problemei curent:
\begin{itemize}
    \item Clasa: \( X_1 \in \{0, 1\} \), unde I = 0, Inferioară = 1;
    \item Sexul: \( X_2 \in \{0, 1\} \), unde Masculin = 0, Feminin = 1;
    \item Vârsta: \( X_3 \in \{0, 1\} \), unde Copil = 0, Adult = 1;
    \item Supraviețuitor: \( Y \in \{0, 1\} \), unde Nu = 0, Da = 1.
\end{itemize}

Setul de date devine:

\begin{tabular}{|c|c|c|c|c|c|}
    \hline
    Indecși & Număr & \( X_1 \) & \( X_2 \) & \( X_3 \) & \( Y \) \\ \hline
    \( [1, 5] \) & \( 5 \) & \( 0 \) & \( 0 \) & \( 0 \) & \( 1 \) \\ \hline
    \( [6, 123] \) & \( 118 \) & \( 0 \) & \( 0 \) & \( 1 \) & \( 0 \) \\ \hline
    \( [124, 180] \) & \( 57 \) & \( 0 \) & \( 0 \) & \( 1 \) & \( 1 \) \\ \hline
    \( [181, 181] \) & \( 1 \) & \( 0 \) & \( 1 \) & \( 0 \) & \( 1 \) \\ \hline
    \( [182, 185] \) & \( 4 \) & \( 0 \) & \( 1 \) & \( 1 \) & \( 0 \) \\ \hline
    \( [186, 325] \) & \( 140 \) & \( 0 \) & \( 1 \) & \( 1 \) & \( 1 \) \\ \hline
    \( [326, 360] \) & \( 35 \) & \( 1 \) & \( 0 \) & \( 0 \) & \( 0 \) \\ \hline
    \( [361, 384] \) & \( 24 \) & \( 1 \) & \( 0 \) & \( 0 \) & \( 1 \) \\ \hline
    \( [385, 1595] \) & \( 1211 \) & \( 1 \) & \( 0 \) & \( 1 \) & \( 0 \) \\ \hline
    \( [1596, 1876] \) & \( 281 \) & \( 1 \) & \( 0 \) & \( 1 \) & \( 1 \) \\ \hline
    \( [1877, 1893] \) & \( 17 \) & \( 1 \) & \( 1 \) & \( 0 \) & \( 0 \) \\ \hline
    \( [1894, 1920] \) & \( 27 \) & \( 1 \) & \( 1 \) & \( 0 \) & \( 1 \) \\ \hline
    \( [1921, 2025] \) & \( 105 \) & \( 1 \) & \( 1 \) & \( 1 \) & \( 0 \) \\ \hline
    \( [2026, 2201] \) & \( 176 \) & \( 1 \) & \( 1 \) & \( 1 \) & \( 1 \) \\ \hline
\end{tabular}

\noindent \textbf{a}.

\( l(w) = \sum_{i=1}^{2201} (y^{(i)} \ln \sigma (w \cdot x^{(i)}) + (1 - y^{(i)}) \ln (1 - \sigma (w \cdot x^{(i)})) ) 
= 5 \cdot 1 \cdot \ln \sigma (w \cdot (1, 0, 0, 0)^T) + 118 \cdot (1 - 0) \cdot \ln (1 - \sigma (w \cdot (1, 0, 0, 1)^T)) 
+ 57 \cdot 1 \cdot \ln \sigma (w \cdot (1, 0, 0, 1)^T) + 1 \cdot 1 \cdot \ln \sigma (w \cdot (1, 0, 1, 0)^T) 
+ 4 \cdot (1 - 0) \cdot \ln (1 - \sigma (w \cdot (1, 0, 1, 1)^T)) + 140 \cdot 1 \cdot \ln \sigma (w \cdot (1, 0, 1, 1)^T) 
+ 35 \cdot (1-0) \cdot \ln (1 - \sigma (w \cdot (1, 1, 0, 0)^T)) + 24 \cdot 1 \cdot \ln \sigma (w \cdot (1, 1, 0, 0)^T)
+ 1211 \cdot (1 - 0) \cdot \ln (1 - \sigma (w \cdot (1, 1, 0, 1)^T)) + 281 \cdot 1 \cdot \ln \sigma (w \cdot (1, 1, 0, 1)^T) 
+ 17 \cdot (1 - 0) \cdot \ln (1 - \sigma (w \cdot (1, 1, 1, 0)^T)) + 27 \cdot 1 \cdot \ln \sigma (w \cdot (1, 1, 1, 0)^T) 
+ 105 \cdot (1 - 0) \cdot \ln (1 - \sigma (w \cdot (1, 1, 1, 1)^T)) + 176 \cdot 1 \cdot \ln \sigma (w \cdot (1, 1, 1, 1)^T) 
= 5 \ln \sigma (w_0) + 118 \ln (1 - \sigma(w_0 + w_3)) + 57 \ln \sigma (w_0 + w_3) + \ln \sigma (w_0 + w_2) 
+ 4 \ln (1 - \sigma (w_0 + w_2 + w_3)) + 140 \ln \sigma (w_0 + w_2 + w_3) + 35 \ln (1 - \sigma (w_0 + w_1)) + 24 \ln \sigma (w_0 + w_1) 
+ 1211 \ln (1 - \sigma (w_0 + w_1 + w_3)) + 281 \ln \sigma (w_0 + w_1 + w_3) + 17 \ln (1 - \sigma (w_0 + w_1 + w_2)) + 27 \ln \sigma (w_0 + w_1 + w_2) 
+ 105 \ln (1 - \sigma (w_0 + w_1 + w_2 + w_3)) + 176 \ln \sigma (w_0 + w_1 + w_2 + w_3) \)

\noindent \textbf{b}.

\textbf{i}.

\( \nabla_w l(w) = \sum_{i=1}^{2201} [ y^{(i)} - \sigma (w \cdot x^{(i)}) ] x^{(i)} 
= 5 [ 1 - \sigma (w_0) ] (1, 0, 0, 0)^T - 118 \sigma (w_0 + w_3) (1, 0, 0, 1)^T + 57 [ 1 - \sigma (w_0 + w_3) ] (1, 0, 0, 1)^T 
+ [ 1 - \sigma (w_0 + w_2) ] (1, 0, 1, 0)^T - 4 \sigma (w_0 + w_2 + w_3) (1, 0, 1, 1)^T + 140 [ 1 - \sigma (w_0 + w_2 + w_3) ] (1, 0, 1, 1)^T 
- 35 \sigma (w_0 + w_1) (1, 1, 0, 0)^T + 24 [ 1 - \sigma (w_0 + w_1) ] (1, 1, 0, 0)^T 
-1211 \sigma (w_0 + w_1 + w_3) (1, 1, 0, 1)^T + 281 [ 1 - \sigma (w_0 + w_1 + w_3) ] (1, 1, 0, 1)^T 
- 17 \sigma (w_0 + w_1 + w_2) (1, 1, 1, 0)^T + 27 [ 1 - \sigma (w_0 + w_1 + w_2) ] (1, 1, 1, 0)^T 
-105 \sigma (w_0 + w_1 + w_2 + w_3) (1, 1, 1, 1)^T + 176 [ 1 - \sigma (w_0 + w_1 + w_2 + w_3) ] (1, 1, 1, 1)^T 
= (711 - 5 \sigma (w_0) - 175 \sigma (w_0 + w_3) - \sigma (w_0 + w_2) - 144 \sigma (w_0 + w_2 + w_3) 
- 59 \sigma (w_0 + w_1) - 1492 \sigma (w_0 + w_1 + w_3) - 44 \sigma (w_0 + w_1 + w_2) - 281 \sigma (w_0 + w_1 + w_2 + w_3), 
508 - 59 \sigma (w_0 + w_1) - 1492 \sigma (w_0 + w_1 + w_3) - 44 \sigma (w_0 + w_1 + w_2) - 281 \sigma (w_0 + w_1 + w_2 + w_3),
344 - \sigma (w_0 + w_2) - 144 \sigma (w_0 + w_2 + w_3) - 44 \sigma (w_0 + w_1 + w_2) - 281 \sigma (w_0 + w_1 + w_2 + w_3), 
654 - 175 \sigma (w_0 + w_3) - 144 \sigma (w_0 + w_2 + w_3) - 1492 \sigma (w_0 + w_1 + w_3) - 281 \sigma (w_0 + w_1 + w_2 + w_3) )^T \)

\textbf{ii}.

Alegem \( j = 2 \):

\( \frac{\partial}{\partial w_2} l(w) = \frac{\partial}{\partial x_2} \ln \sigma (w_0 + w_2) 
+ \frac{\partial}{\partial x_2} 4 \ln (1 - \sigma (w_0 + w_2 + w_3)) + \frac{\partial}{\partial x_2} 140 \ln \sigma (w_0 + w_2 + w_3) 
+ \frac{\partial}{\partial x_2} 17 \ln (1 - \sigma (w_0 + w_1 + w_2)) + \frac{\partial}{\partial x_2} 27 \ln \sigma (w_0 + w_1 + w_2) 
+ \frac{\partial}{\partial x_2} 105 \ln (1 - \sigma (w_0 + w_1 + w_2 + w_3)) + \frac{\partial}{\partial x_2} 176 \ln \sigma (w_0 + w_1 + w_2 + w_3) \)

\end{document}